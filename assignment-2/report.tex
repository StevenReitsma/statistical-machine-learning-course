\documentclass[paper=a4, fontsize=10pt]{scrartcl} % A4 paper and 11pt font size

\usepackage[T1]{fontenc} % Use 8-bit encoding that has 256 glyphs
\usepackage{fourier} % Use the Adobe Utopia font for the document - comment this line to return to the LaTeX default
\usepackage[english]{babel} % English language/hyphenation
\usepackage{amsmath,amsfonts,amsthm} % Math packages
\usepackage{graphicx}
\usepackage[cm]{fullpage}
\usepackage{float}
\usepackage{sectsty} % Allows customizing section commands
\allsectionsfont{\centering \normalfont\scshape} % Make all sections centered, the default font and small caps

\usepackage{fancyhdr} % Custom headers and footers
\pagestyle{fancyplain} % Makes all pages in the document conform to the custom headers and footers
\fancyhead{} % No page header - if you want one, create it in the same way as the footers below
\fancyfoot[L]{} % Empty left footer
\fancyfoot[C]{} % Empty center footer
\fancyfoot[R]{\thepage} % Page numbering for right footer
\renewcommand{\headrulewidth}{0pt} % Remove header underlines
\renewcommand{\footrulewidth}{0pt} % Remove footer underlines
\setlength{\headheight}{13.6pt} % Customize the height of the header

\numberwithin{equation}{section} % Number equations within sections (i.e. 1.1, 1.2, 2.1, 2.2 instead of 1, 2, 3, 4)
\numberwithin{figure}{section} % Number figures within sections (i.e. 1.1, 1.2, 2.1, 2.2 instead of 1, 2, 3, 4)
\numberwithin{table}{section} % Number tables within sections (i.e. 1.1, 1.2, 2.1, 2.2 instead of 1, 2, 3, 4)

\setlength\parindent{0pt} % Removes all indentation from paragraphs - comment this line for an assignment with lots of text

%----------------------------------------------------------------------------------------
%	TITLE SECTION
%----------------------------------------------------------------------------------------

\newcommand{\horrule}[1]{\rule{\linewidth}{#1}} % Create horizontal rule command with 1 argument of height

\title{	
\normalfont \normalsize 
\textsc{Radboud University Nijmegen}  % Your university, school and/or department name(s)
\horrule{0.5pt} \\[0.3cm] % Thin top horizontal rule
\huge Statistical Machine Learning \\ Assignment 2 \\ % The assignment title
\horrule{2pt}  % Thick bottom horizontal rule
}

\author{Steven Reitsma \\ (s4132343)} % Your name

\date{\normalsize\today} % Today's date or a custom date

\begin{document}

\maketitle % Print the title

\section{Sequential learning}
\subsection{Obtaining the prior}
\begin{enumerate}
	\item 
		\begin{align}
			\boldsymbol{\tilde \Lambda} &= \boldsymbol{\tilde \Sigma}^{-1}\\
						   &= \begin{pmatrix}
								   60 &  50 & -48 &  38\\
								   50 &  50 & -50 &  40\\
								  -48 & -50 &  52.4 & -41.4\\
								   38 &  40 & -41.4 &  33.4\\
							  \end{pmatrix}
		\end{align}

		Using the precision matrix we can use equations 2.69, 2.73 and 2.75 from Bishop to compute the mean and covariance of the conditional distribution $p([x_1, x_2]^T \vert x_3 = x_4 = 0)$.

		\begin{align}
			\boldsymbol\Sigma_p &= \boldsymbol\Lambda_{aa}^{-1} \tag{according to 2.73}\\
			\boldsymbol\Lambda_{aa} &= \begin{pmatrix}
								60 & 50 \\
								50 & 50
							\end{pmatrix} \\
			\boldsymbol\Sigma_p &= \begin{pmatrix}
							0.1 & -0.1 \\
   							-0.1 &    0.12 \
   						\end{pmatrix}
		\end{align}

		\begin{align}
			\boldsymbol\mu_p = \boldsymbol\mu_{a} - \boldsymbol\Lambda_{aa}^{-1} \boldsymbol\Lambda_{ab} (\boldsymbol x_{b} - \boldsymbol\mu_{b}) \tag{according to 2.75}\\
		\end{align}
		In this equation, $a$ is the first partition and $b$ is the second partition when partitioned according to equation 2.65.
		\begin{align}
			\boldsymbol\mu_p &= \begin{pmatrix} 1 \\ 0 \end{pmatrix} - \begin{pmatrix} 60 & 50 \\ 50 & 50 \end{pmatrix}^{-1} \begin{pmatrix} -48 & 38 \\ -50 & 40 \end{pmatrix} (\begin{pmatrix} 0 \\ 0 \end{pmatrix} - \begin{pmatrix} 1 \\ 2 \end{pmatrix}) \\
			 &= \begin{pmatrix} 1 \\ 0 \end{pmatrix} - \begin{pmatrix} 0.1 & -0.1 \\ -0.1 & 0.12 \end{pmatrix} \begin{pmatrix} -48 & 38 \\ -50 & 40 \end{pmatrix} (\begin{pmatrix} 0 \\ 0 \end{pmatrix} - \begin{pmatrix} 1 \\ 2 \end{pmatrix}) \\
			 &= \begin{pmatrix} 0.8 \\ 0.8 \end{pmatrix}
		\end{align}
	\item Using the numpy-equivalent of \verb|mvnrnd| I obtained $\boldsymbol\mu_t$.

			\begin{verbatim}
				np.random.multivariate_normal(mu, sigma, 1)
			\end{verbatim}

			returned:

			\begin{equation}
				\boldsymbol\mu_t = \begin{pmatrix} 0.78848608 \\ 0.87686679 \end{pmatrix}
			\end{equation}
	\item
\end{enumerate}

\end{document}